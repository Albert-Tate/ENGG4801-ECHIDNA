\documentclass[12pt,openany,a4paper]{book}
\usepackage{graphics}	% if you want encapsulated PS figures.
\usepackage{titlesec}
\usepackage{textcomp}
\usepackage{float}
\usepackage{adjustbox}
\usepackage{graphicx}
\usepackage{wrapfig}
\usepackage[table]{xcolor}

\titleformat{\chapter}{\normalfont\huge\bf}{\thechapter.}{20pt}{\huge}
% If you use a macro file called macros.tex :
% \input{macros}
% Note: The present document has its macros built in.

% Number subsections but not subsubsections:
\setcounter{secnumdepth}{2}
% Show subsections but not subsubsections in table of contents:
\setcounter{tocdepth}{2}

\pagestyle{headings}		% Chapter on left page, Section on right.
\raggedbottom

\setlength{\topmargin}		{-5mm}  %  25-5 = 20mm
\setlength{\oddsidemargin}	{10mm}  % rhs page inner margin = 25+10mm
\setlength{\evensidemargin}	{0mm}   % lhs page outer margin = 25mm
\setlength{\textwidth}		{150mm} % 35 + 150 + 25 = 210mm
\setlength{\textheight}		{240mm} % 

\renewcommand{\baselinestretch}{1.2}	% Looks like 1.5 spacing.

% Stop figure/tables smaller than 3/4 page from appearing alone on a page:
\renewcommand{\textfraction}{0.25}
\renewcommand{\topfraction}{0.75}
\renewcommand{\bottomfraction}{0.75}
\renewcommand{\floatpagefraction}{0.75}

% THEOREM-LIKE ENVIRONMENTS:
\newtheorem{defn}	{Definition}	% cf. \dfn for cross-referencing
\newtheorem{theorem}	{Theorem}	% cf. \thrm for cross-referencing
\newtheorem{lemma}	{Lemma}		% cf. \lem for cross-referencing

% AIDS TO CROSS-REFERENCING (All take a label as argument):
\newcommand{\eref}[1] {(\ref{#1})}		% (...)
\newcommand{\eq}[1]   {Eq.\,(\ref{#1})}		% Eq.~(...)
\newcommand{\eqs}[2]  {Eqs.~(\ref{#1}) and~(\ref{#2})}
\newcommand{\dfn}[1]  {Definition~\ref{#1}}	% Definition~...
\newcommand{\thrm}[1] {Theorem~\ref{#1}}	% Theorem~...
\newcommand{\lem}[1]  {Lemma~\ref{#1}}		% Lemma~...
\newcommand{\fig}[1]  {Fig.\,\ref{#1}}		% Fig.~...
\newcommand{\tab}[1]  {Table~\ref{#1}}		% Table~...
\newcommand{\chap}[1] {Chapter~\ref{#1}}	% Chapter~...
\newcommand{\secn}[1] {Section~\ref{#1}}	% Section~...
\newcommand{\ssec}[1] {Subsection~\ref{#1}}	% Subsection~...

% AIDS TO FORMATTING:
\newcommand{\teq}[1]	{\mbox{$#1$}}	% in-Text EQuation (unbreakable)
\newcommand{\qed}	{\hspace*{\fill}$\bullet$}	% end of proof

% MATHEMATICAL TEMPLATES:
% Text or math mode:
\newcommand{\half}	{\ensuremath{\frac{1}{2}}}	% one-half
\newcommand{\halftxt}	{\mbox{$\frac{1}{2}$}}	  	% one-half, small
% Math mode only:
% N.B. Parentheses are ROUND; brackets are SQUARE!
\newcommand{\oneon}[1]	{\frac{1}{#1}}		  % reciprocal
\newcommand{\pow}[2]	{\left({#1}\right)^{#2}}  % Parenthesized pOWer
\newcommand{\bow}[2]	{\left[{#1}\right]^{#2}}  % Bracketed pOWer
\newcommand{\evalat}[2]	{\left.{#1}\right|_{#2}}  % EVALuated AT with bar
\newcommand{\bevalat}[2]{\left[{#1}\right]_{#2}}  % Bracketed EVALuated AT
% Total derivatives:
\newcommand{\sdd}[2]	{\frac{d{#1}}{d{#2}}}		    % Short
\newcommand{\sqdd}[2]	{\frac{d^2{#1}}{d{#2}^2}}	    % 2nd ("SQuared")
\newcommand{\ldd}[2]	{\frac{d}{d{#1}}\left({#2}\right)}  % Long paren'ed
\newcommand{\bdd}[2]	{\frac{d}{d{#2}}\left[{#2}\right]}  % long Bracketed
% Partial derivatives (same sequence as for total derivatives):
\newcommand{\sdada}[2]	{\frac{\partial {#1}}{\partial {#2}}}
\newcommand{\sqdada}[2]	{\frac{\partial ^{2}{#1}}{\partial {#2}^{2}}}
\newcommand{\ldada}[2]	{\frac{\partial}{\partial {#1}}\left({#2}\right)}
\newcommand{\bdada}[2]	{\frac{\partial}{\partial {#1}}\left[{#2}\right]}
\newcommand{\da}	{\partial}

% ORDINAL NUMBERS:
\newcommand{\ith}	{\ensuremath{i^{\rm th}}}
\newcommand{\jth}	{\ensuremath{j^{\rm th}}}
\newcommand{\kth}	{\ensuremath{k^{\rm th}}}
\newcommand{\lth}	{\ensuremath{l^{\rm th}}}
\newcommand{\mth}	{\ensuremath{m^{\rm th}}}
\newcommand{\nth}	{\ensuremath{n^{\rm th}}}

% SINUSOIDAL TIME AND SPACE-DEPENDENCY FACTORS:
\newcommand{\ejot}	{\ensuremath{e^{j\omega t}}}
\newcommand{\emjot}	{\ensuremath{e^{-j\omega t}}}

% UNITS (TEXT OR MATH MODE, WITH LEADING PADDING SPACE IF APPLICABLE):
% NB: These have not been tested since being modified for LaTeX2e.
\newcommand{\pack}	{\hspace{-0.08em}}
\newcommand{\Pack}	{\hspace{-0.12em}}
\newcommand{\mA}	{\ensuremath{\rm\,m\pack A}}
\newcommand{\dB}	{\ensuremath{\rm\,d\pack B}}
\newcommand{\dBm}	{\ensuremath{\rm\,d\pack B\pack m}}
\newcommand{\dBW}	{\ensuremath{\rm\,d\pack B\Pack W}}
\newcommand{\uF}	{\ensuremath{\rm\,\mu\pack F}}
\newcommand{\pF}	{\ensuremath{\rm\,p\pack F}}
\newcommand{\nF}	{\ensuremath{\rm\,n\pack F}}
\newcommand{\uH}	{\ensuremath{\rm\,\mu\pack H}}
\newcommand{\mH}	{\ensuremath{\rm\,m\pack H}}
\newcommand{\Hz}	{\ensuremath{\rm\,H\pack z}}
\newcommand{\kHz}	{\ensuremath{\rm\,k\pack H\pack z}}
\newcommand{\MHz}	{\ensuremath{\rm\,M\pack H\pack z}}
\newcommand{\GHz}	{\ensuremath{\rm\,G\pack H\pack z}}
\newcommand{\J}		{\ensuremath{\rm\,J}}
\newcommand{\kg}	{\ensuremath{\rm\,k\pack g}}
\newcommand{\K}		{\ensuremath{\rm\,K}}
\newcommand{\m}		{\ensuremath{\rm\,m}}
\newcommand{\cm}	{\ensuremath{\rm\,cm}}
\newcommand{\km}	{\ensuremath{\rm\,k\pack m}}
\newcommand{\mm}	{\ensuremath{\rm\,m\pack m}}
\newcommand{\nm}	{\ensuremath{\rm\,n\pack m}}
\newcommand{\um}	{\ensuremath{\rm\,\mu m}}
\newcommand{\Np}	{\ensuremath{\rm\,N\pack p}}
\newcommand{\s}		{\ensuremath{\rm\,s}}
\newcommand{\ms}	{\ensuremath{\rm\,m\pack s}}
\newcommand{\us}	{\ensuremath{\rm\,\mu s}}
\newcommand{\V}		{\ensuremath{\rm\,V}}
\newcommand{\mV}	{\ensuremath{\rm\,m\Pack V}}
\newcommand{\W}		{\ensuremath{\rm\,W}}
\newcommand{\mW}	{\ensuremath{\rm\,m\Pack W}}
\newcommand{\ohm}	{\ensuremath{\rm\,\Omega}}
\newcommand{\kohm}	{\ensuremath{\rm\,k\Omega}}
\newcommand{\Mohm}	{\ensuremath{\rm\,M\Omega}}
\newcommand{\degs}	{\ensuremath{\rm^{\circ}}}

% LaTeX run-time type-in command:
%
% \typein{Enter \protect\includeonly{...} command (or just type RETURN):}
%
% Uncommenting this command makes LaTeX prompt you for the \includeonly
% list.  At the prompt
%
%	\@typein=
%
% you type
%
%	\includeonly{chap1,chap2}
%
% to include the files chap1.tex and chap2.tex and omit any others.
% To include every \include file, just hit RETURN.
% If you are running LaTeX from xtexsh, you may need to click the mouse
% in the LaTeX window to position the cursor at the \@typein prompt.

\begin{document}

\frontmatter
% By default, frontmatter has Roman page-numbering (i,ii,...).

\begin{titlepage}
\renewcommand{\baselinestretch}{1.0}
\begin{center}
\vspace*{35mm}
\Huge\bf
	%Low Power Sensor Board for Tracking Lightweight Animals \\
	$\mu$Tracker - Ultra Compact Sensing Platform \\
\vspace{20mm}
\large\sl
		by\\
		Albert Tate
		\medskip\\
\rm
		School of Information Technology and Electrical Engineering,\\
		The University of Queensland.\\
\vspace{30mm}
		Submitted for the degree of\\
		Bachelor of Engineering
		\smallskip\\
\normalsize
		in the field of Electrical and Computer Engineering
		\medskip\\
\large
		November 2015		
\end{center}
\end{titlepage}

\cleardoublepage

\begin{flushright}
	\bf Albert Tate\\ \normalfont
	42910444\\
	67 Swan Street\\
	Gordon Park\\
	QLD 4031\\
	\medskip
\end{flushright}
\begin{flushleft}
  Prof Paul Strooper\\
  Head of School\\
  School of Information Technology and Electrical Engineering\\
  The University of Queensland\\
  St Lucia, Q 4072\\
  \bigskip\bigskip
  Dear Professor Strooper,
\end{flushleft}

In accordance with the requirements of the degree of Bachelor of
Engineering in the School of Information Technology and Electrical Engineering,
I present the following thesis entitled 
\begin{center}
	\emph{``$\mu$Tracker - Ultra Compact Sensing Platform''}
\end{center}  This work was performed under the supervision of Dr Philip Terrill.
I declare that the work submitted in this thesis is my own, except as
acknowledged in the text and footnotes, and has not been previously
submitted for a degree at The University of Queensland or any other
institution.

\begin{flushright}
	Yours sincerely,\\
	\medskip
	\makebox[1.0in]{\hrulefill}\\
	\medskip
	Albert Tate.
\end{flushright}

\cleardoublepage

\chapter{Acknowledgments}

Acknowledge your supervisor, preferably with a few short and specific
statements about his/her contribution to the content and direction of
the project.  If you collaborated with another student, acknowledge
your partner's contribution, including any parts of the thesis of
which s/he was the principal author or co-author; this information can
be duplicated in footnotes to the chapters or sections to which your
partner has contributed.  Briefly describe any assistance that you
received from technical or administrative staff.  Support of family
and friends may also be acknowledged, but avoid sentimentality---or
hide it in the dedication.
%Acknowledge UWA, Craig Freakley for BeeBoard

\cleardoublepage

\chapter{Abstract}

% Yeah really need to like actually redo this.

The use of biologging for remote sensing of animal behavior has had a long history and many novel applications \cite{Coudert05}. A key datatype in recent years has been tri-axial accelerometry \cite{Ran12,Shepard08} which can be used to accurately discern between different movement types and postures in animals or humans \cite{Lamprecht14,Terrill13}. Despite the widespread use of biologging for animal research, most solutions are either very heavy \cite{Marshall07} or, when light, rely on existing infrastructure in the environment like radio base stations \cite{Jurdak13} or cell-phone towers \cite{Bennett11}. In this report a project is proposed to redesign an existing board, the BEE-UTE Board \cite{Freakley13} to be smaller, more efficient and have a longer operating lifetime. This will allow the board to be used on a larger variety of animals and allow for greater time between recaptures of the animal for data retrieval which will have both monetary benefits and less impact on the welfare of the animal. The proposed project will be able to measure light intensity, tri-axial accelerometry, temperature, air pressure and GPS position in a package with a total weight of less than 50 grams including battery and a board size of less than 2.5cm by 2.5cm which will operate at 100Hz sampling rate with selective GPS readings for at least 14 days. The GPS readings will activate when significant activity is measured, calculated by a simple moving average (SMA) on the accelerometer data. The low power functionality will be achieved by a novel power control scheme and the use of power efficient volatile memory, SRAM, to allow much larger bulk writes to non-volatile memory (microSD card).

\tableofcontents

\listoffigures
\addcontentsline{toc}{chapter}{List of Figures}

\listoftables
\addcontentsline{toc}{chapter}{List of Tables}

% If file los.tex begins with ``\chapter{List of Symbols}'':
% \include{los}

\cleardoublepage

\mainmatter
% By default, mainmatter has Arabic page-numbering (1,2,...).


% Chapters may be \include files, each beginning with a line like
%
%	\chapter{Title of chapter}
%
% e.g. if two chapter files were called intro.tex and theory.tex,
% we would say
%
%	\include{intro}
%	\include{theory}

\chapter{Introduction}
Echidnas are small native animals of Australia and New Guinea. The echidna is from the order monotremata meaning, in non technical terms, that it is a mammal that lays eggs. Monotremes are a unique order of animals with only five extant species and very little is known about their physiological response to environment stressors.\\ 

Generally speaking, echidnas struggle with external temperatures greater than about 35 degrees celsius \cite{Brice02}. This suggested that their thermoregulation came from an external source much like cold blooded animals. However recent studies have suggested that this might not be the case and the general assumption is that echidnas employ a combination of both strategies, behavioural and physiological, to manage their body temperature. This has lead to a significant research interest in being able to track these animals in the wild.\\

Due to strict ethics regulations mandating extremely low weights in biologging electronics \cite{Mamm87}, current existing biologging devices find themselves regularly unsuitable for such an application, usually limited by an extremely short battery life. Modern technologies and methods should now allow for a new, significantly smaller, lighter, more robust and more power efficient biologging device to be created. \\

These biologging devices would not be limited to just echidnas however; there are a significant number of animals in the same weight class that are poorly understood. Such a device would allow large studies of predator-prey dynamics in the wild, tracking migratory animals in extreme environments and remote monitoring of unfenced livestock. \\

In light of this scenario, there is a significant research potential in such a device being developed; an incredibly lightweight and power efficient tracker capable of extended operation in harsh environments.

\chapter{Background}
	\section{Echidna Physiology \& Behaivour}
	Echidna thermoregulation is poorly understood. There is no complete consensus on how they do it \cite{Brice02} but current studies suggest it is a combination of behaivoural patterns, like laying in the sun as a cold blooded animal would do, and physiological mechanisms, like any warm blooded creature.\\ 
	
	Their internal body temperature has been measured from as high as 35\textdegree C and as low as 5\textdegree C during hibernation. The echidna will usually seek shelter in hot conditions and is not known to sweat or pant. During autumn and winter, the echidna will generally go into a hibernative state, severely reducing activity. \\
	
	While the echidna are active however, they very seldom have interactions with other creatures. They are generally considered to be solitary creatures. \\
	
	Despite their solitary lifestyle, when they are not in hibernation they can cover significant distances within a single day, swim about as well as most mammals and will dig into ant and termite nests for food. \\
	
	Despite the diet of ants, the echidna can grow up to 6kg in weight for males and around 4.5kg for females. The echidna is only about 30-45cm long in the body which makes it a relatively dense creature. \\

	The echidna is found all around Australia and in the southern mountains of New Guinea and as such, it has managed to survive in a range of climates. It has been found in both snowy regions and dry deserts. They do not have fixed shelter and tend to wander about, which makes population studies difficult. \\
	
	%Due to a lack of suitable biologging platforms, existing studies have generally been done outside of the native habitat, which may have limited the accuracy of any results.
	
	\section{Product Specification}
	In light of this knowledge from the previous section, it is now possible to know what the biologging device should do and why. The key performance characteristic will be operational lifetime as there is an existing product \cite{Freakley13} that otherwise meets the specifications, but only lasts 40 hours in the wild. \\
	
	The device needs to be extremely light to comply with ethics regulations \cite{Mamm87} usually being around 5\% of the body weight of the creature. In addition to this it must be extremely robust, it needs to handle the large temperature swings found in deserts, the enclosure must be waterproof and handle most low speed impacts. \\
	
	In order to properly track any behaviour the echidna may exhibit in order to change its body temperature, the enclosure must be transparent, to allow for light sensing. Rudimentary movement sensing also needs to be used. The device should be able to track absolute position, where appropriate, using a GPS module. Tracking internal temperature can not be reasonably done with a board solution and will be done with another device, the specification and selection of which is out of the scope of this design. However the environment temperature will be sensed. \\
	
	Due to the hibernative nature of the echidna, the onboard firmware should have some rudimentary analysis on the data to allow it to change sampling rates depending on the level of activity, to best extend the battery life. \\
	
	Table \ref{tab:SPEC} quantitatively specifies some of the  other requirements of the design: \\
	\begin{table}[H]
		\centering
		\begin{adjustbox}{max width=\textwidth}
			\rowcolors{2}{white}{lightgray}
			\begin{tabular}{c | c }
				Description & Specification\\
				\hline
				Max system weight & 50g \\
				Max board size & 1500mm$^{2}$  \\
				Operational lifetime & 7 days \\
				Max material cost & \$150 AUD \\
				Robustness & IP67 \\
				Min storage space & 1GB
			\end{tabular}
		\end{adjustbox}
		\caption{Device Specification}
		\label{tab:SPEC}
	\end{table}
	
	\section{Previous Solutions}
		\subsection{BEE-UTE Board}
			In previous studies of echidna movements, the BEE-UTE board was used \cite{Freakley13} which was developed at the University of Queensland (UQ). It was designed as a general base board for biologging. The board was 625mm$^{2}$ and had a 32bit microcontroller onboard as well as sensors for acceleration, magnetic heading, angular acceleration, pressure and temperature. The board also provided headers for future expansion. Further details can be seen in Figure \ref{fig:BEE}. \\
			
			 The board had an incredibly powerful onboard microprocessor and was very expandable whilst retaining a relatively small footprint. However this design philosophy lead to a number of limitations in the context of small animal biologging. The most significant of these limitations is the short battery life, only 40 hours in the field, when outfitted with a GPS module. The onboard processor is overpowered for the application and any broken sensors will cause the system to endlessly poll that sensor, halting data acquisition, although this problem can be solved with a firmware patch. \\
			 
			  A less generalized approach to this specific design problem is needed for this project however this resource will be incredibly valuable reference throughout development. The power management electronics and uSD interface, including power transistors, are all very applicable to this project and will likely be incorporated.
			\begin{figure}[H]
				\centering
				\includegraphics[width=350px]{Figures/BeeBoardDiagram.png}
				\caption{BEE-UTE Board System Diagram}
				\label{fig:BEE}
			\end{figure}
			
		\subsection{Camazotz}
			The Camzotz Board, developed by CSIRO \cite{Jurdak13}, is a small tracking board for fruit bats that is very similar in scope to the project described here but has a few key deviations. The board has onboard sensors for GPS, accelerometry, pressure, temperature and acoustic signals, see Figure \ref{fig:Camazotz} for details. \\
			
			With some very clever duty cycling of peripheral sensors and the use of small solar panels the board is energy neutral, which means the board will operate until device failure; battery life is effectively infinite in most cases. A novel method was used to power the sensors; all the power was drawn directly from I/O ports of the microcontroller which allowed the board to completely turn off certain devices when they weren't being used, a significant energy saving compared to putting these devices in sleep mode. Communication of data was done via radio base stations located in known nesting roosts. \\
			
			Despite this being very similar in scope to this project, there are a few design decisions that ultimately make it unsuitable for this application. The solar panels would interfere with the echidnas attempts to warm up in the sun and the lack of any fixed nest for the echidna makes the radio base station approach not appropriate. \\
			
			The Camazotz board is a good example of miniaturized embedded design as the board itself only weighs 30g. The effective power management and GPS chipset both proved to deliver significant power savings and are both components of the design which will likely be factored into this project. 
			\begin{figure}[H]
				\centering
				\includegraphics[width=5cm]{Figures/CamazotzDiagram.png}
				\caption{System Diagram of Camazotz Board}
				\label{fig:Camazotz}
			\end{figure}		
		\subsection{Crane Tracker}
			The CraneTracker \cite{Bennett11} is a wireless sensor network based platform for tracking Whooping Cranes on their migration from Texas through to Canada. The system is built on an Iris Mote with a ZigBee compliant radio transceiver. The actual devices make use of solar panels for extra power and primarily takes GPS readings.\\
			
			 Fundamentally the communication strategy employed by this study is incompatible with the proposed project; the radios use cellular networks to communicate their data, a luxury not available in the Australian outback. Overall this is a good example of a low weight design but is ultimately focused on solving the problem in a drastically different way. \\
			 
			 There are not many specific design problems solved by this product for the primary application, even though it serves the same purpose. \\ 
			 
			 This device illustrates an underlying issue with the literature and the general progression of the art; a movement has occurred towards 'smart' nodes that are able to communicate with each other and surrounding networks, like the cellular one. Unfortunately in the Australian outback these networks are generally not available as Echidnas are not particularly social or return to the same location, making any radio reliant methods unreliable at best for this application.
			 
		\newpage
		\subsection{CritterCam}
			One of the earlier remote sensing applications was the Crittercam \cite{Marshall07} which was expanded from its original analog video camera to a more complete digital system including a variety of data sensors. \\
			
			The Crittercam is primarily a video camera for marine animals. The newest version (Gen V) includes accelerometers, magnetometers, pressure, temperature and flowmeter sensors. All of this is sampled data is written to a MultiMedia Card (MMC), a predecessor to the SDcard. \\
			
			 Some of the biggest successes of the Crittercam are its incredible robustness and ability to actively monitor battery power and change modes when the battery is critical (when low battery is detected, the device will become a radio beacon to aid retrieval). However the reliance on video data is a relic from the earlier models and, compared to the large power cost, is of little use in a lot of research situations.\\
			 
			  Unfortunately the weight of the unit is 1.1kg which is almost heavier than some of the animals the sensor board proposed in this report could be deployed on. This is one of the earliest commercial remote sensing applications and appears here mostly for completeness as the style of approach to the problem, with no reliance on external infrastructure, is aligned with the goals for this project however miniaturizing the technology to such an extent has historically been very difficult.\\
			  
			  The device does show a number of novel design decisions which are appropriate for this project. Since it has been such a long running project, the system engineering has been extensively honed and all the space available has been used well. 
	\newpage
	\section{Technical Background}
		Many useful lessons can be taken from the previous products that exist in the art however these alone do not create an entire solution. This section will provide some technical background that is necessary to both fulfill the requirements set out by this project and to adequately justify a number of design decisions. This is not by any means an exhaustive background and some familiarity with embedded systems and electronics design is assumed. \\
		
		 This section only exists to inform the reader of the relevant technologies, uses and other considerations that need to be taken into account to design an embedded system for this application. For further details on exact component selection and system architecture see section \ref{sec:components}. 
	%Remember: This section is about technology choices, not specific parts
		\subsection{Microcontrollers}
		A microcontroller is a general purpose computing device that can make basic calculations and communicate with external peripherals like sensors and memory modules. In general these can consume significant amounts of power with the PIC32 on the BEE-UTE Board \cite{Freakley13} drawing as much as 30mA in an active state \cite{PIC32}. \\
		
		In the context of this application, relatively little processing power is needed, as there is only a small amount of computations occurring onboard. What is required is an extremely low current draw, small physical size without relying on Ball Grid Array (BGA) footprints and a bevy of peripheral communication abilities. 
		\subsection{Movement Sensing}
			Detecting animal movement usually requires the synergy of several sensors. Classifying movement states and localizing position are typically done separately. Movement is usually sensed by measuring acceleration of which there are two main approaches; measuring the acceleration along an axis (accelerometer), or measuring rotational acceleration across axes (gyroscope). Gyroscopes typically consume more power than accelerometers by a factor of 10 or more \cite{InvenMPU9150} so for this, low power, application, accelerometers will be used to measure movement. \\
			
			Accelerometers are devices which measure acceleration along an axis. The vast majority of accelerometers have 3 orthogonal axes of measurement. Tri-axial accelerometry has been used as a primary data source in the past, as movement states can be classified from just this data alone \cite{Ran12, Shepard08} and can be much more accurate and less invasive than equivalent measuring methods for determining movement states \cite{Terrill13, Lamprecht14}. 
		\subsection{GPS Module}
			Global Positioning System (GPS) modules are devices which interact with satellites to determine their exact position. Implementation specifics vary but the fundamental concept is for the module to establish communication with four or more satellites using an external antenna and the align the internal clock with them to localize its position. \\
			
			This requires fairly high level floating point arithmetic which is usually provided by the module itself. These satellite fixes can take as long as 30 seconds. A typical GPS module can draw as much as 50mA in active mode \cite{Carroll10}.
			
		\subsection{Peripheral Communications}
			In embedded systems there are two major protocols for communicating between devices that might have separate clocks. These methods are known as Serial Peripheral Interface (SPI) and Inter-Integrated Circuit (I2C). \\
			
			I2C is a simple, two wire interface while SPI is more complex. It requires 3 wires plus an extra wire for every device attached. In general I2C is a slower interface than SPI although that shouldn't be a limiting factor in this application. \\
			
			Since size is a significant requirement in this project, care should be taken to choose peripheral devices that operate on I2C so as to minimize the amount of printed circuit board (PCB) traces. 
			
		\subsection{Volatile \& Non Volatile Memories}
			There are two major kinds of memory in electronic systems; volatile and non volatile each with their own power/use tradeoffs. Volatile memory is memory which loses the saved data whenever the power is disconnected whereas non volatile memory will retain memory permanently. \\
			
			The major design tradeoff between these two methods is energy consumption; it is very inefficient to write bursts of small data to a non volatile memory \cite{Sandisk} however the cost is reduced significantly when writing to volatile memory. \\
			
			The system will require some form of non volatile memory however the solution should have both forms of memory on board, to minimize frequency of writing to non volatile memory.
			
		\subsection{Power Supply}
			Most electronic devices are designed with very specific input voltages to power them; small deviations can damage the device permanently. Battery voltages also decrease the longer they are used. Because of this, power supplies are used to regulate the voltage being supplied. \\
			
			There are two major kinds of power supplies being used; linear and switching regulators \cite{TI2011}. Linear regulators use a transistor in the active region of operation to behave as a variable resistor, reducing the voltage. Because of this design, linear regulators can only ever output a lower voltage than their input. This design also incurs a significant amount of energy loss as the input and output currents are the same but the voltage changes significantly. \\
			
			Switching regulators are more complex and usually use a reference waveform and a comparator to vary the duty cycle of an output stage transistor which is then fed into a filtering circuit to stabilize the voltage into a specified DC value. They can also step up voltages from lower ones to higher ones. Because of this added complexity and the generation of the reference waveform, they tend to consume a small amount of current, called the quiescent current, even when not supplying energy to the circuit but despite this are generally much more efficient than linear regulators, typically around 90\%. 
			
		\subsection{Battery Chemistries}
			There are a number of different battery chemistries being used in embedded system applications, the three most common being: Alkaline, Lithium-Ion and Lithium-Polymer. \\
			
			Alkaline cells rely on the interaction between zinc and mangnanese (IV) oxide. These cells are typically not rechargeable and can be quite heavy as they are typically produced inside a steel case. They also have a problem with high current draws; the capacity of the battery decreases with increasing load current. \\
			
			Lithium-Ion is probably the most common battery type seen in robotics. They have a large capacity to weight ratio and are rechargeable. However they are most commonly produced in a hard plastic case which increases the weight significantly. \\
			
			Lithium-Polymer or, more correctly, lithium-ion polymer is the same chemistry as a lithium ion cell, but packaged in a soft case. This makes them substantially lighter than other chemistries per unit charge as shown in Figure \ref{fig:BAT} but makes them susceptible to damage. \\
			
			\begin{figure}[H]
				\centering
				\includegraphics[width=14cm]{Figures/EDC.png}
				\caption{Energy Density of Various Chemistries \cite{icc}}
				\label{fig:BAT}
			\end{figure}		
	
			This damage is a significant factor for this project. When lithium, and a few other chemistries, cells are damaged, they can sometimes undergo a state called thermal runaway, where the entirety of their energy is converted to heat in a few seconds. This can be quite violent and appropriate protections need to be taken into account to stop this from happening to any batteries placed on an animal.

\chapter{Design Overview}
	NOTE: Many circuit schematics to go in here, might be a bit bare figure wise at the moment. Also missing some explicit discussion on circuitry specifics but we'll see how we're going for space when more is complete. \\ \noindent\makebox[\linewidth]{\rule{\paperwidth}{0.4pt}}
	\noindent\makebox[\linewidth]{\rule{\paperwidth}{0.4pt}}
	\section{Hardware} \label{sec:components}
		The general approach to component selection for this project has been a careful balancing act of size, power consumption and in some cases weight. Choosing the best devices to start from was absolutely essential in designing a system to meet the specifications. A few of these were selected over a few revisions; initially the project was implemented on small custom PCB's in order to test functionality. \\
		
		The designed system has utilised a few key elements from the solutions looked at previously in order to accomplish the stated specification. The microcontroller can turn peripheral devices on and off directly from the IO pins and there is a multilevel cache made of volatile and non volatile memories to get the best battery life out of the device. This will enable the microcontroller to spend most of the period sleeping, only moving to wake state to make a short series of measurements. An architecture overview is shown in Figure \ref{fig:HWD}.
		
		\begin{figure}[H]
			\centering
			\includegraphics[width=11cm]{Figures/HWD.png}
			\caption{Hardware Diagram}
			\label{fig:HWD}
		\end{figure}		
		\newpage
		
		\subsection{Microcontroller}
		The microcontroller was considered to be one the key choices with regards to achieving the operating lifetime. In the field there are a number of competing device lines in super low power applications such as Microchip's nanoWatt, Atmels picoPower and Texas Instruments MSP430 series. \\
		
		All these device lines could theoretically be used for the device when only considering power usage, they all have active mode currents in the range of $\mu$A/MHz. \\
		
		Atmel's picoPower line is quite young and expensive, which is not very attractive in this range of products however it was the most familiar. Using one of the development boards quickly revealed that, while functional, the microcontrollers were quite susceptible to soft faults and would often reset without notice. This combined with the lack of extensive documentation and an overly bloated framework lead to the adoption of an alternative solution being sought. \\
		
		The two competing microcontrollers left belonged to the MSP430 series and the PIC24F series. Eventually PIC24F was chosen over the MSP430 as it was found to be more user friendly and had one key feature; remappable peripherals. The PIC24F series can move the peripheral fucntions, like I2C, SPI, UART and others, onto any available I/O pin. This flexibility allowed for the most freedom when performing PCB placement as there would rarely be a situation where a pin was stuck in a tricky place. \\
		
		The PIC24FJ128GA306 was the final microcontroller chosen \cite{PIC24} and boasted current draws as low as 150 $\mu$A/MHz in active mode and around 400 nA in sleep mode with a real time clock enabled. The device is also relatively small, just 10mm by 10mm by 1mm, making it ideal for these applications. There is no BGA version of this exact chip but there is another in the PIC24FJ128GA310 family but due to the increased pin count is the same size.
		
		\newpage
		\subsection{Sensors}
		Sensors can have a noticeable impact on battery life. Some sensors, such as gyroscopes, can draw current in the range of mA. For most sensors, absolute accuracy is the ultimate design goal and so power consumption typically suffers. Luckily for most sensors, there exist some, relatively expensive, solutions that are highly configurable in order to customize the tradeoff between accuracy and power consumption. \\
		
			\subsubsection{Accelerometry}
			There are a number of exciting new entries into the low power accelerometry chip market. Most notable is the Bosh BMI160 \cite{bosch15} which has ultra low operational currents of less than a mA with both the accelerometer and gyroscope enabled. \\
			
			Unfortunately the BMI160 could not be used as the data would not be compatible with the existing set of data from the InvenSense MPU9150 \cite{InvenMPU9150} on the BEE-UTE board. The MPU9150 is an integrated inertial measurement unit (IMU) with an onboard accelerometer, thermometer, gyroscope and magnetometer. The MPU9150 draws 900uA with the accelerometer enabled, which is more than enough for this application as the device will spend the vast majority of its time powered down. 
			\subsubsection{Pressure \& Temperature}
			The temperature sensor onboard the MPU9150 is not accuract enough for the levels of precision required and does not have an onboard pressure sensor strangely enough. In order to get the most accurate reading of either, pressure and temperature should be sampled together. \\
			
			The Measurement Specialties MS5637 \cite{MEASPRESSURE} was chosen for this application as it boasted a super low power consumption, was relatively cheap and has a small footprint. There were a number of others in its class but a lot of them had onboard processing, the MS5637 instead requires the microcontroller to read in the calibration words and use them to convert the unitless measurements into metric values. 
			\subsubsection{Light Sensor}
			Most digital light sensors required a comparatively large amount of power to operate and had relatively large footprints. In light of this a completely analog light sensor was chosen, the Avago APDS-9005-020 \cite{AVAGOLIGHT}. \\
			
			The purpose of the sensor was to primarily tell if the echidna was resting in the shade or out in direct sunlight so metric values weren't important and with the PIC24F's onboard 10 bit ADC, precision wasn't a problem for this sensor. 
			\subsubsection{GPS Localisation}
			The GPS was probably the hardest sensor to decide upon. It had to be everything GPS sensors aren't; small, power efficient and fast operating. Eventually the u-blox MAX6 \cite{ubloxGPS} was chosen. \\
			
			The MAX6 can draw currents as small as 12mA during a 1Hz update rate, as much as five times smaller than the norm for a GPS. This however comes with a lot of drawbacks, namely price. The device is \$60 USD and can, at the time of writing, only be bought in Australia directly from the manufacturer. \\
			
			These concessions had to be made for this project however, as the GPS alone would require more current than every other component combined otherwise.
			\subsubsection{Battery Meter}
			With non volatile data storage, corruption of memory, especially in the presence of a under voltage event, becomes more likely. The battery meter will help the application track the battery voltage allowing it to shutdown safely when the battery is in danger of an under voltage event. \\
			
			The battery meter chosen was the Maxim Integrated MAX17040 \cite{max17} mainly due to the low cost and size. %More?
			
		\newpage
		\subsection{Data Storage}
		One of the key design decisions to this project is to have a multilevel cache with volatile and non volatile memories shown in Figure \ref{fig:MLD}. The first layer would be provided by the mircocontroller itself, with internal RAM. The next layer would be two of Microchip's 23LCV1024 1Mbit serial SRAM \cite{MICRAM}. The final layer would be a regular, off the shelf, $\mu$SD card. \\
				
				\begin{wrapfigure}{r}{0.4\textwidth}
					\centering
					\includegraphics[width=5cm]{Figures/MLD.png}
					\caption{Memory Architecture}
					\label{fig:MLD}
				\end{wrapfigure}
		
		$\mu$SD cards can draw incredibly large currents in write mode \cite{Sandisk} and as such, can not be relied upon for primary storage; the cost of small writes is simply too great. Instead the approach taken was to store as much as possible on the microcontroller then write this out in bursts to the volatile SRAM. Once this was full, an entire 2Mbit write could occur to the $\mu$SD card, greatly reducing the power consumption. \\
		
		With the selection of this serial SRAM and generating around 160 bytes of sensor readings every second (at a 1 Hz sampling rate) the $\mu$SD card will only need to be written to every 26 minutes. Without the serial SRAM, this would have been reduced to only 2 minutes. \\
		
		\subsection{Power Supply}
		The power supply is one area where any mistakes have a cumulative effect on system power consumption. Quiescent currents and inefficiencies can become very significant in the face of relatively minor design errors. \\
		
		In order to maintain relatively safe with these issues, and to not fix what isn't broken, the same components as appeared on the BEE-UTE board were chosen. These devices are both Texas Instruments branded: the BQ24072 battery charging and power path management IC \cite{TIUSB} and the TPS63030DSK switching buck-boost regulator \cite{TIBUCK}. \\
		
		The BQ24072 allows for a user to charge the battery onboard or optionally power the device from the USB bus. This enables a potential fixed position application or an alternate, 5 volt source to power the device, although battery voltage tracking would not be available. In addition, the BQ24072 also has a number of thermal protection and short-circuit protections which will prove themselves essential in the rugged environment of the Australian outback. \\
		
		The TPS63030 is a fairly standard buck-boost switching regulator with a relatively flat efficiency curve of around 80\% for most of the operating range of currents. There is also around a 50uA quiescent current, which is very small for the price range of the device. The TPS63030 retails for approximately \$6 AUD and most power supplies that can provide higher efficiencies across the incredibly large range this device will have will be too costly for this project. \\
		
		
		%TALK ABOUT SELECTION OF PASSIVES
		Programming the output voltage on the TPS63030 is remarkably simple; Texas Instruments provide all the relevant equations and best practices for layouts and component specifications. The feedback voltage is fixed to around 500mV inside the device, although in practice this varied slightly depending on the load attached, which resulted in some fine tuning on the finished product. See Appendix [!REF] for detailed calculations and information about selection of passives and their specifications.
		
		\subsection{Batteries}
		Now that the components are selected, it is appropriate to draft a power budget to better inform the size of battery needed. It has already been decided to go with lithium polymer batteries, which are significantly lighter per unit charge than most other types which should allow maximum flexibility. \\
		
		Table \ref{tab:PWR} shows the required battery sizes for various power management strategies. The three strategies are no control (N.C.), basic control (B.C.) and advanced control (A.C.) and refer to the extent to which duty cycling certain components is performed. Note that an analysis like this will not factor in things like inefficiencies, quiescent currents,  self discharge etc so these results more serve to provide an upper bound on performance. Actual performance could be as low as half of these predicted values.\\
		
		\begin{table}[H]
			\centering
			\begin{adjustbox}{max width=\textwidth}
				\rowcolors{2}{white}{lightgray}
				\begin{tabular}{c | c | c | c | c | c }
					Device & Active Current & Sleep Current & N.C. Duty Cycle & B.C. Duty Cycle & A.C. Duty Cycle\\
					\hline
					PIC24F          & 150$\mu$A & 410nA  & 100\% & 20\%  & 10\%  \\
					MAX6 GPS        & 12mA      & -      & 100\% & 100\% & 5\%   \\
					Light Sensor    & 400$\mu$A & -      & 100\% & 100\% & 10\%  \\
					Accelerometer   & 925$\mu$A & 3$\mu$A& 100\% & 20\%  & 10\%  \\
					Pressure Sensor & 0.6$\mu$A & -      & 100\% & 100\% & 10\%  \\
					$\mu$SD Card    & 50mA      & -      & 25\%  & 10\%  & 5\%   \\
					\hline
					\bf BATTERY REQUIRED & - & - & 8.72A-h & 7.6A-h & 1.3A-h\\
				\end{tabular}
			\end{adjustbox}
			\caption{Battery requirements by management method}
			\label{tab:PWR}
		\end{table}		
		
		It is immediately clear that some heavy management of duty cycle is necessary for this project to work correctly. A battery of about 8A-h would be incredibly expensive and far too heavy for the application. 1.2A-h lithium polymer batteries are relatively common, come in a variety of sizes and weigh about 20 grams. Alternatively, to improve the battery life, a 2A-h battery can be connected which weighs 40 grams.
		
		\subsection{Connectors}
		A frequently overlooked component in system design is user end connectors being of poor quality or being of a simple 0.1 inch pitch unkeyed connector. After seeing many projects go up in flames it was apparent that small keyed connectors were needed. \\
		
		A standard microUSB type B plug was used for the battery charging, due to its small size and being a common cable throughout the world. A JST connector was used for the battery input, to prevent any users from accidentally connecting it backwards and also allowing the system to operate with any single cell lithium based battery which will expand the possible applications significantly as the user won't be required to resolder anything to change the battery. 
	\newpage
	\section{Firmware}
		The approach to firmware was primarily iterative; an initial plan was laid out early in the project's life but this was repeatedly updated and refined in the face of new information or implementation subtleties. The choice to go without an embedded operating system was not taken lightly; an embedded operating system can save the developer having to delve into the specifics of a microcontrollers architecture. Ultimately it was decided that this benefit was not worth the inevitable lack of flexibility and operating overhead. \\
		
		Fundamentally the system had a fairly straightforward task: during each sampling period, turn on all the sensors, give them time to settle, take the measurements then push the data out to the multilevel cache. The original high level flowchart is shown in Figure \ref{fig:FWD}
		
		\begin{figure}[H]
			\centering
			\includegraphics[width=11cm]{Figures/FWD.png}
			\caption{Firmware Flowchart}
			\label{fig:FWD}
		\end{figure}		

		\newpage
		\subsection{Design \& Implementation} %Working title
		The design was always centered around readability and, while that may be highly subjective, a large effort was put forward to name that. For structure and flow control the OpenBSD kernel maintainers style guide \cite{BSDstyle} was used, as it is well suited to such a low level environment. \\
		
		Gonna work on this part more once I've cleaned up the source code a little, its still a bit messy.
		\subsection{Functionality}
		\subsection{Key Excerpts}
	\newpage
	\section{Manufacture}
		Manufacture of miniaturised designs is always a challenge and can prove difficult both in producing PCB artwork and physically constructing the device. Since this is a prototype, everything had to be hand constructed which has somewhat limited miniturisation; namely in that BGA based footprints cannot be reasonably used. 
		
		\subsection{PCB Design}
			The largest concern during PCB design was size, as the size of the device will limit the possible cases used and, as the boards and the copper within are quite heavy, can have a noticeable impact on final system weight. \\
			
			%Talk about board physical dimensions?
			The board was designed to have roughly the same physical dimensions as a standard USB stick but this had to be increased slightly to 20mm x 60mm for practicality reasons. \\
			
			The PCB artwork was developed using the Altium Designer suite of tools and, by the last iteration, was comprised of 6 layers; four signal layers and two plane layers. This was a necessary design decision to not only shield components from electromagnetic interference (EMF) from components on the other side but to also act as a heatsink for the large transient currents that could be experienced by the system. \\
			
			There was a substantial effort wherever possible to reduce EMF however on such a small board, some concessions had to be made. The power section of the board, with its heavy switching noise, was separated away from the sensors and the microcontroller. The serial SRAM will spend most of its time idle, so during the sampling period there should be relatively little EMF and this was exploited by placing the bulk of the sensors on the opposite side of the board. \\
			
			The largest concession made was with the GPS module. It is placed directly underneath the microcontroller. This was the primary motivation for the two plane layers and is further alleviated by the small airgap between the PCB and the microcontroller. Despite this a completely isolated section of board was allowed for the antenna so the actual performance should not be hindered in any noticeable way. \\
			
			Wherever appropriate, the smallest passives were used. This meant 0402 footprints for every passive except for the power inductors and the user LEDs, for specification and visibility reasons respectively. The picture shown in Figure \ref{fig:PCB} shows the final PCB with the control circuitry highlighted in green, the sensing circuitry in blue and the power supply circuitry in red. \\

		\begin{figure}[H]
			\centering
			\includegraphics[width=9cm]{Figures/PCB.PNG}
			\includegraphics[width=9cm]{Figures/PCBMarkup.PNG}
			\caption{Final PCB}
			\label{fig:PCB}
		\end{figure}		
			
			%Approach taken
			%Picture of bare PCB with separation pointed out
			%Passives (0402)
		\subsection{Bill of Materials}
		TODO: Must format nicely
		\subsection{Construction}
		There are a number of practical difficulties with construction that need to be fixed in subsequent iterations but in the current state, this needs to be done to be able to build the device. \\
		
		Some devices on the top side have a thermal pad directly underneath them. This helps with a stable ground connection and reduces self heating but it also prevents the use of any glues. Since these components are QFN footprint; they can not be reliably soldered by hand. The issue arises on the reverse side where there is a QFN style footprint on the antenna which is only rated to go through the oven once. \\
		
		This means that the only way to reliably solder the device is to use non-leaded solder paste on the top side, which has a higher melting point, and apply the top side components first. Then use leaded solder, with a much lower melting point, on the reverse side. This will stop the solder joints remelting and any components falling off the device during the soldering process.
		
		%Maybe talk about iterations here?
	\section{Integration}
	The purpose of this design is to be used in a wide variety of entirely uncontrolled, and very hostile, environments. As such for the device to survive for longer than a few moments it must be protected by a case. This case would need to be waterproof and shock resistant as well as quite small. \\
	
	Initially several options were drafted to construct a case out of polyethylene from a 3d printed mold fitted with an off the shelf o-ring to prevent moisture seeping in. This case would have weighed less than 5 grams and would have vastly improved the size of battery connected to the device. \\
	
	Unfortunately due to a number of practicality and time issues a completely off the shelf solution was sought. The ideal case in this scenario would be a small specimen container commonly found in biology labs. The screw top would allow for a relatively waterproof design to start with, and a small o-ring would improve it significantly. A picture of possible enclosures is shown in Figure \ref{fig:CASE}.
	
	\begin{figure}[H]
		\centering
		\includegraphics[width=9cm]{Figures/CASE.JPG}
		\caption{Candidate cases}
		\label{fig:CASE}
	\end{figure}			

	%Talk about cases here
	NOTE: Maybe talk about which cases could and could not be used? \& Talk about actually putting everything together and the challenges within (Definitely do this, need to detail PCB iterations and breadboard pictures)\\
	
		\begin{figure}[H]
			\centering
			\includegraphics[width=6cm]{Figures/finalpt1.png}
			\caption{Revision 2 of the Design}
			\label{fig:FIN}
		\end{figure}	
		
	\noindent\makebox[\linewidth]{\rule{\paperwidth}{0.4pt}}
	\noindent\makebox[\linewidth]{\rule{\paperwidth}{0.4pt}}
		
\chapter{Discussion}

	\section{Validation of Design}
		The design and construction process for this project was iterative, as most engineering designs tend to be, and went through several versions before some concrete testing could be done. \\
		
		There were three main targets to achieve with this project: \\
		\begin{enumerate}
			\item Echidna appropriate volume and total weight less than 50 grams
			\item An enclosure rated at IP67 or better
			\item An operational lifetime of a week or better
		\end{enumerate} 
		Due to a number of development issues involving time constraints and component sourcing not all of the functionality could be completed but the design does satisfy these three targets or provide significant evidence that they would be satisfied.\\
		\subsection{Size \& Weight}
		The final PCB produced, which was very similar to \ref{fig:FIN} but with a rotated battery input and some miscellaneous fixes to usability, weighs 6.55 grams.
		
		The final container used, due to an oddly shaped battery, is the red capped container in Figure \ref{fig:CASE}. This container on its own weighs 27.65 grams. \\
		
		The battery, as mentioned before, weighs 22.6 grams. This brings the total system weight to 56.85 grams (measured, not calculated). This is slightly more than the initial requirements but a more appropriate case could alleviate this issue significantly. This would allow for any animal around 1.1kg or heavier to be fitted out with this device \cite{Mamm87}. \\
		
		It is worth noting that by using the 2A-h battery, the total system weight is 72.70 grams. This is significantly heavier than planned but is a good demonstration of the flexibility of design. This would allow for any animal around 1.5kg and heavier to be fitted out with this device. \\
		
		Both of these implementations would be able to be placed on most echidnas, but care would have to be taken to not place it on an echidna too light. It may not achieve the stated goals exactly as described but the weight is within the acceptable bounds for the primary application.
		\subsection{Enclosure}
		TODO: IP67 Tests
		\subsection{Operational Lifetime}
		
			\begin{figure}[H]
				\centering
				\includegraphics[width=14cm]{Figures/input.png}
				\caption{$\mu$Tracker Input Current \& Average DC Equivalent}
				\label{fig:input}
			\end{figure}	
			
		\subsection{Sensor Accuracy}
		TODO: Accuracy Tests
		\subsection{PCB}
		The final PCB submitted was almost feature complete except for two small errors. Firstly the $\mu$SD card header footprint was reversed, and had to be bodged manually and the control pin for providing power to the $\mu$SD card was connected to an input only pin of the PIC24F. Fortunately this pin was right next to an unused output so a small solder bridge was all that was needed to correct it. \\
		
		Since submission these two issues have been fixed and there has been a separation of power grounds and the rest of the circuit, to minimise switching noise. \\
		
	\section{Incomplete Work}
	Unfortunately, due to a combination of several issues, some features and sections of the project could either not be finished, or not be tested. Where otherwise mentioned, every entry into this section implicitly  should appear in Section \ref{sec:FUTURE}, "Future Work".
	
	TODO: Flesh these out naturally
	\subsubsection{GPS Module}
	\subsubsection{Proof of $\mu$SD Card}
	\subsubsection{Full Lifetime Measurement}
	
	\section{Specification Comparison}
	\section{Future Work} \label{sec:FUTURE}

\chapter{Conclusion}

\ldots 

\chapter{User Manual}

\ldots

\appendix

% Chapters after the \appendix command are lettered, not numbered.
% Setting apart the appendices in the table of contents is awkward:

\newpage
\addcontentsline{toc}{part}{Appendices}
\mbox{}
\newpage

% The \mbox{} command between two \newpage commands gives a blank page.
% In the contents, the ``Appendices'' heading is shown as being on this
% blank page, which is the page before the first appendix.  This stops the
% first appendix from be listed ABOVE the word ``Appendices'' in the
% table of contents.

% \include appendix chapters here.

\chapter{Appendix}

Appendices are useful for supplying necessary details or explanations
which do not seem to fit into the main text, perhaps because they are
too long and would distract the reader from the central argument.
Appendices are also used for program listings.

Notice that appendices are ``numbered'' with capital letters, not
numerals.  When the \verb+\appendix+ command in
\LaTeX~\cite[p.\,175]{lamport} is used with the \texttt{book} document
class, it causes subsequent chapters to be treated as appendices.

\chapter{Program listings}

\section{First program}

Some initial explanatory notes may precede the listing.

\section{Second program}

\section{Etc.}

\chapter{Companion disk}

If you wish to make some computer files available to your examiners,
you can list and describe the files here.  The files can be supplied
on a disk and inserted in a pocket fixed to the inside back cover.

The disk will not be needed if you can specify a URL from which the
files can be downloaded.

\cleardoublepage


\begin{thebibliography}{99}
	\addcontentsline{toc}{chapter}{Bibliography}
	\bibitem{Coudert05}
	Yan Ropert-Coudert, Rory P. Wilson
	\emph{Trends and Perspectives in Animal-Attached Remote Sensing},
	Frontiers in Ecology and the Environment,
	Vol. 3, No. 8 (Oct. 2005), pp. 437-444
	
	\bibitem{Freakley13}
	Craig Freakley
	\emph{BEE-UTE Board System Reference Manual},
	University Of Queensland, June 2014
	
	\bibitem{Jurdak13}
	Raja Jurdak et al
	\emph{Camazotz: Multimodal Activity-Based GPS Sampling},
	Information Processing in Sensor Networks,
	ISBN: 978-1-4503-1959-1, pp. 67-68
	
	\bibitem{Bennett11}
	William P Bennett et al
	\emph{CraneTracker: A Multi-Modal Platform for Monitoring Migratory Birds on a Continental Scale},
	The ACM 17th Annual International Conference on Mobile Computing and Networking, 2011.
	
	\bibitem{Marshall07}
	Greg Marshal et al
	\emph{An Advanced Solid-state Animal-borne Video and Environmental Data-logging Device('CRITTERCAM') for Marine Research},
	Marine Technology Society Journal,
	Vol. 41, Issue 2 (June 2007), pp. 31-38
	
	\bibitem{Mamm87}
	The American Society of Mammologists (1987) \emph{Acceptable Field Methods of Mammalogy Preliminary guidelines prepared by the American Society of Mammalogists},
	Journal of Mammalogy Supp. Vol 68, No. 4. November p.13.
	
	\bibitem{Brice02}
	Peter H. Brice et al
	\emph{Heat tolerance of short-beaked echidnas (Tachyglossus aculeatus) in the field},
	University of Queensland, Jan 2002
	
	\bibitem{Ran12}
	Nathan, Ran et al
	\emph{Using tri-axial acceleration data to identify behavioral modes of free-ranging animals: general concepts and tools illustrated for griffon vultures.},
	The Journal of Experimental Biology (2012), 215(6), 986$-$996. doi:10.1242/jeb.058602
	
	\bibitem{Shepard08}
	Emily Shepard et al
	\emph{Identification of animal movement patterns using tri-axial accelerometry.},
	Endang Species Res 10:47-60 (2008)
	
	\bibitem{Lamprecht14}
	Marnie Lamprecht et al
	\emph{Multisite accelerometry for sleep and wake classification in children},
	Physiol. Meas. (2014) doi:10.1088/0967-3334/36/1/33
	
	\bibitem{Terrill13}
	Philip Terrill et al
	\emph{Measuring leg movements during sleep using accelerometry: Comparison with EMG and piezo-electric scored events},
	University of Queensland (July 2013) doi:10.1109/EMBC.2013.6611134
	
	\bibitem{Sandisk}
	Sandisk Corporation. (2007, Jun.) \\
	\emph{SanDisk SD Card Product Family - Product Manual.} [Online].\\ http://media.digikey.com/pdf/Data\%20Sheets/M-Systems\%20Inc\%20PDFs/SD\%20Card\%20Prod\%20Family\%20OEM\%20Manual.pdf
	
	\bibitem{TIUSB}
	Texas Instruments. (2015, Mar) \emph{bq2407x 1.5-A USB-Friendly Li-Ion Battery Charger and Power-Path Management IC} [Online],
	http://www.ti.com/lit/ds/symlink/bq24072.pdf
	
	\bibitem{TIBUCK}
	Texas Instruments. (2012, Mar) 
	\emph{High Efficiency Single Inductor Buck-Boost Converter with 1-A Switches} [Online],
	http://www.ti.com/lit/ds/symlink/tps63030.pdf
	
	\bibitem{PIC24}
	Microchip Technology Inc. (2011) 
	\emph{PIC24FJ128GA310 Family - Reference Manual} [Online],
	http://ww1.microchip.com/downloads/en/DeviceDoc/39996f.pdf
	
	\bibitem{PIC32}
	Microchip Technology Inc. (2008)
	\emph{PIC32MX Family - Reference Manual} [Online], http://ww1.microchip.com/downloads/en/DeviceDoc/PIC32MX\_Datasheet\_v2\_61143B.pdf
	
	\bibitem{TI2011}
	Texas Instruments. (2011)
	\emph{Linear and Switching Voltage Regulator Fundamental Part 1} [Online], http://www.ti.com/lit/an/snva558/snva558.pdf
	
	\bibitem{AVAGOLIGHT}
	Avago Technologies. (2007, Jan) 
	\emph{APDS-9005 Miniature Surface-Mount Ambient Light Photo Sensor} [Online],
	http://www.avagotech.com/docs/AV02-0080EN
	
	\bibitem{MEASPRESSURE}
	Measurement Specialties. (2013, Feb)
	\emph{MS5637-02BA03 Low Voltage Barometric Pressure Sensor}
	[Online],
	http://www.farnell.com/datasheets/1756129.pdf
	
	\bibitem{InvenMPU9150}
	InvenSense Inc. (2013, Sep)
	\emph{MPU-9150 Product Specification Revision 4.3} [Online],
	http://www.invensense.com/mems/gyro/documents/PS-MPU-9150A-00v4\_3.pdf
	
	\bibitem{MICRAM}
	Microchip Technology Inc. (2012)
	\emph{1 Mbit SPI Serial SRAM with Battery Backup and SDI Interface} [Online],
	http://ww1.microchip.com/downloads/en/DeviceDoc/25156A.pdf
	
	\bibitem{ubloxGPS}
	Swiss u-blox (2012)
	\emph{MAX-6 u-blox 6 GPS Modules - Data Sheet} [Online],
	https://www.u-blox.com/images/downloads/Product\_Docs/MAX-6\_DataSheet\_\%28GPS.G6-HW-10106\%29.pdf
	
	\bibitem{Carroll10}
	Aaron Carroll et al \emph{An Analysis of Power Consumption in a Smartphone}.
	USENIX Annual Technical Conference 2010
	
	\bibitem{NXPI2C}
	NXP Semiconductors (2014, Apr)
	\emph{I2C-bus specification and user manual} [Online],
	http://www.nxp.com/documents/user\_manual/UM10204.pdf
	
	\bibitem{bosch15}
	Bosch Sensortech (2015)
	\emph{Small, low power inertial measurement unit} [Online], https://ae-bst.resource.bosch.com/media/products/dokumente/bmi160/BST-BMI160-DS000-07.pdf
	
	\bibitem{icc}
	ICC Nexergy (2015)
	\emph{Comparison of Energy Densities for Various Battery Chemistries} [Online], http://www.iccnexergy.com/battery-systems/battery-energy-density-comparison/
	
	\bibitem{max17}
	Maxim Integrated (2012)
	\emph{Compact, Low-Cost 1S/2S Fuel Gauges} [Online], https://datasheets.maximintegrated.com/en/ds/MAX17040-MAX17041.pdf
	
	\bibitem{BSDstyle}
	OpenBSD (2015)
	\emph{style — Kernel source file style guide (KNF) - Manual Page} [Online], http://www.openbsd.org/cgi-bin/man.cgi/OpenBSD-current/man9/style.9?query=style\&sec=9
	
\end{thebibliography}

\end{document}